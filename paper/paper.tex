%\documentclass[aps,prl,preprint,groupedaddress]{revtex4-1}
%\documentclass[aps,prl,preprint,superscriptaddress]{revtex4-1}
\documentclass[aps,prl,reprint,groupedaddress]{revtex4-1}
\usepackage{graphicx}
% You should use BibTeX and apsrev.bst for references
% Choosing a journal automatically selects the correct APS
% BibTeX style file (bst file), so only uncomment the line
% below if necessary.
%\bibliographystyle{apsrev4-1}

\begin{document}

% Use the \preprint command to place your local institutional report
% number in the upper righthand corner of the title page in preprint mode.
% Multiple \preprint commands are allowed.
% Use the 'preprintnumbers' class option to override journal defaults
% to display numbers if necessary
%\preprint{}

%Title of paper
\title{The Magnetorotational Instability Prefers Three Dimensions}

% repeat the \author .. \affiliation  etc. as needed
% \email, \thanks, \homepage, \altaffiliation all apply to the current
% author. Explanatory text should go in the []'s, actual e-mail
% address or url should go in the {}'s for \email and \homepage.
% Please use the appropriate macro foreach each type of information

% \affiliation command applies to all authors since the last
% \affiliation command. The \affiliation command should follow the
% other information
% \affiliation can be followed by \email, \homepage, \thanks as well.
\author{Dedalus}
\author{Andrew Swan}
% \email[]{Your e-mail address}
%\homepage[]{Your web page}
%\thanks{}
%\altaffiliation{}
\affiliation{}

%Collaboration name if desired (requires use of superscriptaddress
%option in \documentclass). \noaffiliation is required (may also be
%used with the \author command).
%\collaboration can be followed by \email, \homepage, \thanks as well.
%\collaboration{}
%\noaffiliation

\date{\today}

\begin{abstract}
The magnetorotational instability shows growth rates in three dimensions where the two dimensional problem shows nothing. NOTHING!
\end{abstract}

% insert suggested PACS numbers in braces on next line
\pacs{}
% insert suggested keywords - APS authors don't need to do this
%\keywords{}

%\maketitle must follow title, authors, abstract, \pacs, and \keywords
\maketitle

% body of paper here - Use proper section commands
% References should be done using the \cite, \ref, and \label commands

\section{Introduction}
\label{sec:intro}

\section{Method}
\label{sec:method}

We solve the linearized magnetohydrodynamic equations in a rotating, shearing background. The Navier-Stokes equation is in a fairly standard form,
\begin{equation}
  \label{eq:ns}
  \frac{D \mathbf{v}}{Dt} - f \hat{z} \times \mathbf{v} + S v_x \hat{z} + \mathbf{\nabla}{p} + \nu \mathbf{\nabla} \times \mathbf{\omega} = 0,
\end{equation}
where $\omega = \mathbf{\nabla} \times \mathbf{v}$ is the vorticity and $S$ is the background shear rate. We write the induction equation in terms of the $x$ component of the magnetic field,
\begin{equation}
  \label{eq:Bx}
  \frac{D B_x}{Dt} - B_0 \partial_z v_x + \eta (\partial_y J_z - \partial_z J_y) = 0
\end{equation}
and the $x$ component of the current density,
\begin{equation}
  \label{eq:Jx}
  \frac{D J_x}{Dt} - B_0 \partial_z \omega_x + S \partial_z B_x - \eta \nabla^2 J_x = 0
\end{equation}
We explicitly enforce the divergence constraint on the incompressible velocity field $v$ and the magnetic field,
\begin{equation}
  \label{eq:divu}
  \mathbf{\nabla} \cdot \mathbf{v} = \mathbf{\nabla} \cdot \mathbf{B} = 0.
\end{equation}
We consider the behavior of the MRI in a vertically ($z$) and horizontally ($y$) periodic channel of width $L_x = \pi$. The walls at $x = \pm L_x/2$ are perfectly conducting and stress-free. The MRI is a weak-field instability; in the invisicd case 

We solve equations~(\ref{eq:ns})--~(\ref{eq:divu}) by assuming harmonic perturbations in $y$ and $z$, $f = \hat{f}(x) e^{i(k_y y + k_z) + \gamma t}$. This reduces to an eigenvalue problem in $x$, which we solve using the \emph{Dedalus} pseudospectral framework. 

\section{Results}
\label{sec:results}

\begin{figure}[h]
  \includegraphics[width=0.95\columnwidth]{growth_rates.png}
  \caption{Growth rate as a function of $k_z$ and $k_y$. The maximum growth rate is far from $k_y = 0$, indicating the MRI is most unstable in a fully three-dimensional configuration.}
  \label{fig:growth_rate}
\end{figure}
% If in two-column mode, this environment will change to single-column
% format so that long equations can be displayed. Use
% sparingly.
%\begin{widetext}
% put long equation here
%\end{widetext}

% figures should be put into the text as floats.
% Use the graphics or graphicx packages (distributed with LaTeX2e)
% and the \includegraphics macro defined in those packages.
% See the LaTeX Graphics Companion by Michel Goosens, Sebastian Rahtz,
% and Frank Mittelbach for instance.
%
% Here is an example of the general form of a figure:
% Fill in the caption in the braces of the \caption{} command. Put the label
% that you will use with \ref{} command in the braces of the \label{} command.
% Use the figure* environment if the figure should span across the
% entire page. There is no need to do explicit centering.

% \begin{figure}
% \includegraphics{}%
% \caption{\label{}}
% \end{figure}

% Surround figure environment with turnpage environment for landscape
% figure
% \begin{turnpage}
% \begin{figure}
% \includegraphics{}%
% \caption{\label{}}
% \end{figure}
% \end{turnpage}

% tables should appear as floats within the text
%
% Here is an example of the general form of a table:
% Fill in the caption in the braces of the \caption{} command. Put the label
% that you will use with \ref{} command in the braces of the \label{} command.
% Insert the column specifiers (l, r, c, d, etc.) in the empty braces of the
% \begin{tabular}{} command.
% The ruledtabular enviroment adds doubled rules to table and sets a
% reasonable default table settings.
% Use the table* environment to get a full-width table in two-column
% Add \usepackage{longtable} and the longtable (or longtable*}
% environment for nicely formatted long tables. Or use the the [H]
% placement option to break a long table (with less control than 
% in longtable).
% \begin{table}%[H] add [H] placement to break table across pages
% \caption{\label{}}
% \begin{ruledtabular}
% \begin{tabular}{}
% Lines of table here ending with \\
% \end{tabular}
% \end{ruledtabular}
% \end{table}

% Surround table environment with turnpage environment for landscape
% table
% \begin{turnpage}
% \begin{table}
% \caption{\label{}}
% \begin{ruledtabular}
% \begin{tabular}{}
% \end{tabular}
% \end{ruledtabular}
% \end{table}
% \end{turnpage}

% Specify following sections are appendices. Use \appendix* if there
% only one appendix.
%\appendix
%\section{}

% If you have acknowledgments, this puts in the proper section head.
%\begin{acknowledgments}
% put your acknowledgments here.
%\end{acknowledgments}

% Create the reference section using BibTeX:
\bibliography{basename of .bib file}

\end{document}
%
% ****** End of file apstemplate.tex ******

