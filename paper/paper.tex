% PRL Length limits
% Displayed Math 	The word equivalent for displayed math is 16 words per row for single-column equations. Two-column equations count as 32 words per row.
% Figures 	To estimate the word equivalent for figures use the figure’s aspect ratio (width / height). The estimate is [(150 / aspect ratio) + 20 words] for single-column figures, and [300 / (0.5 * aspect ratio)] + 40 words for double-column figures.

% Fig 1: A = w/h = 2. Double column, words = 340
% Fig 2: A = 4/3. Single column, words = 133
% Fig 3: A = 1. Single column, words = 170
% Fig 4: A = 0.5 Single column, words = 320
% equations: 8 * 16 = 128

% current word count: 2057 + 128 + 340 + 133 + 170 + 320
% limit: 3750
\documentclass[aps,prl,reprint,superscriptaddress]{revtex4-1}
\newcommand{\SSC}{S/S_{c}}
\newcommand{\Prm}{\mbox{\textit{Pm}}}
\newcommand{\mnras}{Monthly Notices of the Royal Astronomical Society}
\newcommand{\apjs}{The Astrophysical Journal Supplement Series}
\usepackage{graphicx}

\begin{document}

\title{The Magnetorotational Instability Prefers Three Dimensions}

\author{Jeffrey~S.~Oishi}
\email[]{joishi@bates.edu}
\affiliation{Bates College Department of Physics and Astronomy, Lewiston, ME 04240}

\author{Geoffrey~M.~Vasil}
\affiliation{University of Sydney School of Mathematics and Statistics, Sydney, NSW, Australia}
\author{Morgan Baxter}
\affiliation{Bates College Department of Physics and Astronomy, Lewiston, ME 04240}
\author{Andrew Swan}
\affiliation{Faculty of Mathematics, Cambridge University, Cambridge, United Kingdom}
\author{Keaton~J.~Burns}
\affiliation{Center for Computational Astrophysics, Flatiron Institute, New York, NY 10010}
\affiliation{Massachusetts Institute of Technology Department of Physics, Cambridge, MA 02139}
\author{Daniel~Lecoanet}
\affiliation{Princeton Center for Theoretical Science and Princeton University Department of Astrophysical Sciences, Princeton, NJ 08544}
\author{Benjamin~P.~Brown}
\affiliation{University of Colorado Laboratory for Atmospheric and Space Physics and Department of Astrophysical and Planetary Sciences, Boulder, CO 80309}

\date{\today}

\begin{abstract}
The magnetorotational instability (MRI) occurs when a weak magnetic field destabilizes a rotating electrically conducting fluid with an inwardly increasing angular velocity shear.
The MRI is considered essential to the operation of astrophysical accretion disks, with the central object supplying a Keplerian rotation profile.
The MRI should also exist for the internal shear layers of stars.
Stellar-interior shears can take a wide range of profiles, including near-critical, but with negligible dissipation. 
Here, we show that the fastest growing unstable modes of an ideal magnetofluid are three dimensional provided the shear rate, $S$, is close to the two-dimensional onset value, $S_c$.
Three-dimensional modes are unstable down to $S \approx 0.10 S_c$, and dominate the two-dimensional modes until $S \approx 2.05 S_c$.
Significant numbers of fast three dimensional modes remain well past $2.05 S_{c}$. 
These finding are significant in three ways. 
First, evidence from weakly nonlinear theory suggests that the MRI saturates by pushing the shear rate to its critical value. 
This can happen for systems, like stars, that can rearrange their angular velocity profiles.
Second, the non-normal character and large transient growth of the MRI should be important anytime strong three-dimensionality exists.
Finally, three-dimensional growth suggest the possibility of direct dynamo action driven from the linear instability itself.
We demonstrate that three dimensional modes dominate the dynamics for shear profiles relevant to the Solar interior and at the very low magnetic Prandtl numbers relevant to liquid metal laboratory experiments.
\end{abstract}

\pacs{}
\maketitle

The magnetorotational instability (MRI) is extremely important in astrophysical fluid dynamics.
By changing the stability criterion for differentially rotating flows from a negative angular \emph{momentum} gradient to a negative angular \emph{velocity} gradient, a weak magnetic field allows the Keplerian velocity profile due to a central point mass to drive turbulence \citep[e.g.][]{1998RvMP...70....1B,2010RSPTA.368.1607J}.
This discovery explained the ubiquitous accretion onto compact objects at rates compatible with observations, and may also influence the formation of planets \citep[e.g.][]{2007Natur.448.1022J}.
In disks, the gravitational field dominates the local plasma dynamics, and thus the MRI cannot significantly affect the background shear; it must saturate by other means \citep{2018MNRAS.474.3451X}.
However, stars and liquid metal Taylor-Couette experiments have differential rotation profiles driven by much weaker stresses.
Where the MRI is active in these flows, it can saturate by pushing the background shear close to critical \citep{2015RSPSA.47140699V,2017ApJ...841....1C,2017ApJ...841....2C}, analogous to convection mixing entropy.
Stellar interiors have extremely high fluid and magnetic Reynolds numbers, but may potentially operate at or near the critical shear rate for the MRI.
This critical shear is set by the background magnetic field strength for axisymmetric perturbations in a channel of a given size.
Despite the extensive literature on the accretion disk case of strong shear and high Reynolds numbers and the liquid metal literature on weak shear and lower Reynolds numbers, even the linear dynamics of the MRI are not well studied in the weak shear, high Reynolds number regime.

Here, we investigate the stability of three dimensional perturbations near the two dimensional critical shear rate $S_c$ for a nearly invisicd, ideal MHD flow.
We find that the first destabilized modes are three-dimensional, and thus can act as a dynamo even in the absence of secondary instability.
These results strongly suggest that the non-normality of the linear operator driving the MRI are always important, even when axisymmetric modes dominate.

We solve the linearized magnetohydrodynamic equations in rotating plane-Couette geometry.
This corresponds to a Cartesian frame rotating with angular frequency $\Omega$ and a linear background shear, $V_{y} = Sx$. 
We cast the Navier-Stokes equation in a fairly standard form,
\begin{equation}
  \label{eq:mhd}
  \frac{D \mathbf{v}}{Dt} + f \hat{z} \times \mathbf{v} + S v_x \hat{z} + \mathbf{\nabla}{p} + \nu \mathbf{\nabla} \times \mathbf{\omega} = B_{0} \partial_{z} \mathbf{b},
\end{equation}
where
\begin{equation}
\omega = \mathbf{\nabla} \times \mathbf{v}, \quad \text{and} \quad \frac{D}{Dt} \ = \ \partial_{t} + S x \partial_{y}
\end{equation}
The main input parameters are the Coriolis parameter, $f = 2 \Omega$; the background shear rate, $S = dV_{y}/dx < 0$;  the vertical magnetic field $B_{0}$ (in Alfv\'{e}n units $\mu_{0} \rho_{0} = 1$); and the horizontal domain width $-d/2 \le x \le d/2$.
Accretion disk modelling usually considers the Rossby number $q = -S/f < 1$; $q=1$ corresponds to purely hydrodynamical Rayleigh unstable shear.
For all calculations here, we use $q = 3/4$ (Keplerian).
We write the induction equation in terms of the $x$ component of the magnetic field,
\begin{equation}
  \label{eq:Bx}
  \frac{D b_x}{Dt} - B_0 \partial_z v_x + \eta (\partial_y j_z - \partial_z j_y) = 0
\end{equation}
and the $x$ component of the current density ($j_{x} = \partial_{y}b_{z} - \partial_{z} b_{y}$),
\begin{equation}
  \label{eq:Jx}
  \frac{D j_x}{Dt} - B_0 \partial_z \omega_x + S \partial_z B_x - \eta \nabla^2 j_x = 0
\end{equation}
We explicitly enforce the divergence constraint on the incompressible velocity field $v$ and the magnetic field,
\begin{equation}
  \label{eq:divu}
  \mathbf{\nabla} \cdot \mathbf{v} = \mathbf{\nabla} \cdot \mathbf{b} = 0.
\end{equation}
We consider the behavior of the MRI in a vertically ($z$) and horizontally ($y$) periodic channel of width $d = \pi$. The boundary conditions are impenetrable stress-free and perfectly conducting; $v_{x} = \omega_{y} = \omega_{z} = b_{x} = \partial_{x}j_{x} = 0$. The MRI is a weak-field instability; in the inviscid, ideal case the critical shear rate for instability in \textit{two dimensions} is
\begin{equation}
  \label{eq:Sc}
  S_c = -\frac{\pi^2 B_{0}^2}{f d^2}.
\end{equation}
We use $\SSC$ as our instability control parameter. 

We assume harmonic perturbations in $y$ and $z$, (e.g. pressure) $p = \hat{p}(x) e^{i(k_y y + k_z z) + \sigma t}$ . 
We use a complex-valued growth rate $\sigma = \gamma + i\omega$ with $\gamma, \omega$ both real. 
The system reduces to a $\sigma$-eigenvalue problem of 10 first-order ODEs in $x$ with Dirichlet boundary conditions.
We pose and solve the system in the same formulation as equations~(\ref{eq:mhd}-\ref{eq:divu}) using the \emph{Dedalus} framework. 

Our interest is in ideal ($\eta = 0$), inviscid ($\nu = 0$) conditions.
However, we set $\eta=\nu=10^{-5}$ to avoid critical layers in for the \textit{stable} solution branch.
We confirmed our results are insensitive to the presence of small diffusion. 
For each $(k_y, k_z)$ pair, we solve using $n_x = 128$ modes; all our results are identical at double the resolution\footnote{All code used in this paper can be found at \protect\url{https://github.com/jsoishi/mri\_prefers\_3d}}.
\begin{figure*}[ht]
  \includegraphics[width=\textwidth]{fig_1.pdf}
  \caption{Growth rates for three-dimensional MRI modes. a) growth rates $\gamma$ over a grid of $k_y$ and $k_z$ for four values of supercriticality $\SSC$. The gray contour highlights $\gamma = 0$; the red dots indicate the fastest growing mode. At $\SSC = 0.640$, there are no two dimensional modes. b) Growth rate vs $k_y$ for $k_z = 0.259$ at $\SSC = 1.002$ (highlighted by the blue line in a). The orange line gives the asymptotic result from equation~(\ref{eq:asymp}) and the blue line are the numerical results. The asymptotic form is valid as $\SSC \to 1$ and $k_y \ll 1$. c) The full discrete spectrum of the MRI for $(k_y, k_z) \simeq (0.263, 0.447)$. The unstable mode is plotted in orange, its stable complex conjugate is blue, and all other modes are gray.}
  \label{fig:growth_rate}
\end{figure*}

Our two main results are: (I) When the first two-dimensional mode becomes unstable, there already exist three-dimensional modes with positive growth rate. (II) When the criticality becomes sufficiently large, the fastest-growing mode becomes purely two-dimensional. These results are pertinent in two ways. First, the MRI contains a substantial ``Goldilocks regime" with possible direct dynamo action, and this regime likely most applies to stellar interiors. Second, our results accord with well-established results for accretion disks that expect primary two-dimensional linear modes.
Figure~\ref{fig:growth_rate}a shows the growth rates for four values of $\SSC$. 
At $\SSC = 1.02$, the maximum growth rate ($\gamma/S = 0.
093$) happens at $(k_y, k_z) \simeq (0.263, 0.447)$.
The first panel of figure~\ref{fig:growth_rate}a shows $\SSC = 0.640$, which contains only three-dimensional instability.
The overall critical shear for three-dimensional modes is $S_{c,3D} \simeq 0.102 S_c$.

While surprising, the MRI's preference for three-dimensional modes can be predicted analytically.
We analyze equations~(\ref{eq:mhd}-\ref{eq:divu}) with the shear asymptotically close to the critical two-dimensional rate, and compute the leading-order correction to the growth rate from three-dimensional effects assuming $S = S_c - \epsilon \sqrt{R}$, $k_{z} \sim \mathcal{O}(\epsilon)$, $\sigma \sim k_{y} \sim \mathcal{O}(\epsilon^{2})$, 
\begin{equation}
  \label{eq:asymp}
\sigma^{2} =  \frac{q\,B_{0}^{2}}{(q+1)} \left[ k_z^2 \left( R^{2} - \frac{d^2}{\pi^{2}} k_z^2 \right)  +  \Upsilon \, k_{y}^{2}  \right] +  \ldots,
\end{equation}
where
\begin{equation}
\Upsilon \ = \ \frac{\left(6+\pi ^2\right) q+\pi^2-6}{12 }\  \approx \ 1.31 \ \  \text{for}  \ \  q = \frac{3}{4}.
\end{equation}
The first term in equation~(\ref{eq:asymp}) results from the two-dimensional calculation.
The second term is positive definite: it \emph{always} leads to enhanced growth rates when $k_y \neq 0$.
Of course, it ultimately leads to ultraviolet divergence in the absence of higher-order effects.

Figure~\ref{fig:growth_rate}b compares the numerical growth rates for $\SSC=1.002$ between $0 \le k_y \le 0.2$ to the asymptotic approximation, showing good agreement where the latter is valid.
Figure~\ref{fig:growth_rate}c shows the full spectrum for $\SSC = 1.02$.
The plot shows a purely growing/decaying complex-conjugate pair (orange/blue dots on the real axis) consistent with equation~(\ref{eq:asymp}).
The other stable modes (gray dots) are rotationally modified Alfv\'{e}n waves found in left- and right-going pairs, consistent with the analytic predictions for two-dimensional stability calculations.

It is useful to examine the eigenvectors for these three dimensional modes.
Because we are only interested in the \emph{unstable} modes, which have $\omega = 0$, critical layers do not form.
We therefore do not need viscosity or resistivity to regularize the solutions.
In order to test that our non-ideal solutions are sufficiently close to the ideal case we are interested in, we reformulated equations~(\ref{eq:mhd}--~(\ref{eq:divu})) without viscous and resistive terms and solved the corresponding eigenvalue problem.
The ideal MHD system is second order in $x$ and thus only two boundary conditions can be set.
We choose impenetrable ($v_x = 0$) conditions on both walls in this case.
Figure~\ref{fig:eigvec} shows the eigenvectors for the most unstable mode at $\SSC = 1.02$ using ideal MHD. 
The tilted structures of $v_x$ and $v_y$ as well as $b_x$ and $b_y$, show that the eigenmodes have non-trivial Reynolds and Maxwell stresses.
The eigenfunctions are indistinguishable from those at finite resistivity and viscosity, lending additional confidence that our other results are sufficiently in the ideal regime.

\begin{figure}[h!]
  \centering
  \includegraphics[width=\columnwidth]{eigvecs_xy_run_11_ideal_single_mode.pdf}
  \caption{Eigenvectors of the velocity and magnetic field perturbations for the most unstable mode when $\SSC = 1.02$. This calculation is \emph{ideal}: $\eta = \nu = 0$; calculations at $\eta = \nu = 10^{-5}$ are indistinguishable.  The top row shows $v_x$, $v_y$, $v_z$ (red is negative, blue positive); the bottom row shows $b_x$, $b_y$, $b_z$ (purple is negative, orange positive). As these are linear calculations, the amplitudes are arbitrary.}
  \label{fig:eigvec}
\end{figure}

Numerical simulations aimed at understanding the MRI in accretion disks consistently show axisymmetric modes dominating the early evolution of the MRI before breaking down into 3D MHD turbulence \citep{1995ApJ...440..742H,2018ApJ...853..174H,2019ApJS..241...26D}. 
These ``channel modes'' are exact non-linear solutions for $x$-shearing-periodic domains, and can only saturate via parasitic shear instabilities \citep{1994ApJ...432..213G}.
We use impenetrable, stress-free boundary conditions that are more applicable to stellar interiors. 
But even in disks, any kind of finite radial extent will cutoff the unbounded growth of channel modes. 
We reconcile our results with earlier simulations by showing that the MRI indeed prefers two dimensions for the larger values of criticality found in disk simulations. 
Figure~\ref{fig:phi} shows the phase angle $\phi = \arctan k_y/k_z$ of the fastest growing mode as function of $\SSC$.
Above $\SSC \gtrsim 2.05$ $\phi=0$, indicating that axisymmetric modes have the fastest growth rates.
For the fiducial run in \citet{1996ApJ...464..690H}, $\SSC \simeq 4.84$ (most works use similar values).
Thus, we predict axisymmetric modes should dominate the linear dynamics for parameters studied in prior numerical simulations.
\begin{figure}[h!]
  \includegraphics[width=\columnwidth]{phi_vs_ssc_grid.pdf}
  \caption{Angle $\phi$ of mode with respect to $z$ axis vs $\SSC$. The inset shows the transition from axisymmetric modes to three dimensional modes as $\SSC$ decreases below $2.05$. The shaded region indicates stability to all perturbations; the stability limit for three dimensional modes is $\SSC \simeq 0.102$.}
  \label{fig:phi}
\end{figure}

Even for $\SSC> 2.05$ there are significant swaths of unstable three-dimensional modes with growth rates comparable to the maximum (figure~\ref{fig:growth_rate}a).
Also, non-normality generically accompanies non-axisymmetry in shearing systems \citep[see][for a discussion relevant to the MRI]{1992MNRAS.255P..25K}.
In non-normal systems, transient amplification can occur for stable modes, and can even cause turbulence.
Therefore significant three dimensionality likely implies important non-normal behavior near onset.

The second implication of the three dimensional modes is of considerably more importance.
These modes produce dynamo action in the linear regime, as evidenced by the presence of non-trivial eigenmodes for all three components of the magnetic field.
This is of considerable significance, as it leads to the possibility of a laminar MRI dynamo operating directly in regions of weak shear.
MRI dynamos have been studied in a number of different contexts \citep{2007PhRvL..98y4502R,2011ApJ...740...18O,2015PhRvL.114h5002S}, but all except one have focused on the turbulent dynamo. The lone exception, \citet{2016MNRAS.462..818B}, is a numerical study that discovered the exponential growth of mean magnetic fields during the linear growth phase of the MRI far from stability.
Our work explains this result in terms of purely linear dynamics:
non-axisymmetric MRI unstable modes drive the exponential growth of magnetic fields.

\begin{figure}[h!]
  \centering
  \includegraphics[width=\columnwidth]{run_47_output_growthrates.pdf}\\
  \includegraphics[width=\columnwidth]{run_49_output_growthrates.pdf}
  \caption{(top) Growth rates for q=0.1 relevant to the Solar interior. (bottom) Growth rates for MRI near onset at liquid metal-like diffusivities, $\nu=10^{-6}$, $\eta = 0.1$.}
  \label{fig:other_params}
\end{figure}
Finally, we vary two important parameters:
$q$ and the magnetic Prandtl number $\Prm = \nu/\eta$.
Figure~\ref{fig:other_params} shows the growth rates for $\SSC = 1.02$.
The upper panel shows $q = 0.1$ holding all other parameters equal to their fiducial values.
This corresponds a Rossby number relevant to the Sun, and it shows very similar growth rates to the Keplerian cases presented above, $\gamma_{max}/S \simeq 0.1$.
Again, MRI remains dominated by three dimensional modes near onset.
The lower panel shows $\nu = 10^{-6}$ and $\eta = 0.1$ with all other parameters equal to their fiducial. 
This is $\Prm = 10^{-5}$ close to those found in liquid metal experiments.
At this $\Prm$, the MRI is only slightly supercritical; this is very similar to the experimental conditions of the Princeton MRI experiment \citep{2002JFM...462..365G}.
Our results are for the rotating Plane Couette geometry\footnote{This is equivalent to Taylor-Couette flow in the limit that the gap between cylinders $R_2 - R_1 \to 0$.} with highly idealized boundary conditions.
Nevertheless, it is very suggestive that near onset in both shear and resistivity at very low $\Prm$, the MRI is \emph{only} unstable to three dimensional modes.

Our results show that that whenever the MRI is near its critical shear values,  three-dimensional modes grow faster than their better-studied two dimensional counterparts.
While we have focused on the Keplerian case ($q=0.75$), we have also demonstrated that the effect is robust and occurs at Rossby numbers $q=0.1$ relevant to the Sun.
There are several important future directions this work suggests.
First, the Sun possesses two internal shear layers, the tachocline and the near surface shear layer.
The latter in particular, though unstable to convection, may also host slower MRI-driven dynamics.
In order to further elucidate this processes, understanding the non-linear saturation of the linear instability presented here is crucial, as is its robustness to perturbations driven by the fast convective dynamics.
Second, our results at low $\Prm$ suggest that three-dimensional MRI modes may be the easiest to excite in liquid metal experiments.
Followup work including more complex boundary conditions and full, cylindrical Taylor-Couette geometries are required to determine if experiments should consider the signatures of these non-axisymmetric modes.
Our prior work on the axisymmetric MRI \citep{2017ApJ...841....1C,2017ApJ...841....2C} shows that differences in linear dynamics between rotating Plane Couette and Taylor-Couette geometries are small in that case; whether or not such a result carries over to non-axisymmetric modes remains to be seen.


\begin{acknowledgments}
We would like to thank Steve Tobias for helpful conversations on this work.
JSO acknowledges funding from NASA LWS grant No. NNX16AC92G and Research Corporation Scialog Collaborative Award (TDA) ID\#24231. Computations were performed on the \emph{Leavitt} cluster at the Bates College High Performance Computing Center.
Morgan Baxter acknowledges support from NASA LWS grant No. NNX16AC92G.
\end{acknowledgments}

\bibliography{mri}

\end{document}

