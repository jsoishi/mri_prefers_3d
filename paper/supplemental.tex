\documentclass[aps,prl,preprint
,superscriptaddress]{revtex4-1}\newcommand{\SSC}{S/S_{c}}
\newcommand{\mnras}{Monthly Notices of the Royal Astronomical Society}
\newcommand{\apjs}{The Astrophysical Journal Supplement Series}
\usepackage{graphicx}
\usepackage{amssymb}
\usepackage{amsmath}

% You should use BibTeX and apsrev.bst for references
% Choosing a journal automatically selects the correct APS
% BibTeX style file (bst file), so only uncomment the line
% below if necessary.
%\bibliographystyle{apsrev4-1}

\newcommand\Beq{\begin{eqnarray}} 
\newcommand\Eeq{\end{eqnarray}}

\newcommand{\Eq}[1]{Eq.~(\ref{#1})}
\newcommand{\Eqs}[2]{Eqs.~(\ref{#1})--(\ref{#2})}
\newcommand{\eq}[1]{eq.~(\ref{#1})}
\newcommand{\eqs}[2]{eqs.~(\ref{#1})--(\ref{#2})}

\newcommand{\Prob}[2]{\mathbb{P}[\, #1 \, | \, #2 \, ] }
\newcommand{\Pro}[1]{\mathbb{P}[\, #1 \, ] }

\newcommand{\inv}[1]{\frac{1}{#1}}
\newcommand{\pd}[1]{\partial_{#1}}

\newcommand{\Ra}{R}
\newcommand{\Ve}{\mbox{\textit{Ve}}}
\newcommand{\eps}{\varepsilon}
\newcommand{\del}{\delta}
\newcommand{\e}[1]{\eps^{#1}}
\newcommand{\f}{2\Omega}

\newcommand{\n}{\hat{n}}


\newcommand{\dat}{\cdot}
\newcommand{\dd}[1]{\,\mathrm{d}{#1}}
\newcommand{\cc}{\mathrm{c.c.}}
\renewcommand{\vec}[1]{\boldsymbol{#1}}
\newcommand{\grad}{\nabla}
\renewcommand{\div}{\nabla \dat}
\newcommand{\Tr}{\text{Tr}}


\newcommand{\pren}[1]{ \left(#1\right) }
\newcommand{\brac}[1]{\left[\,#1\,\right]}

\newcommand{\ave}[1]{\big<#1\big>}

\renewcommand{\l}{\ell}

\newcommand{\txt}[1]{\quad \text{#1} \quad }

\begin{document}

% Use the \preprint command to place your local institutional report
% number in the upper righthand corner of the title page in preprint mode.
% Multiple \preprint commands are allowed.
% Use the 'preprintnumbers' class option to override journal defaults
% to display numbers if necessary
%\preprint{}

%Title of paper
\title{Supplemental Information for: The Magnetorotational Instability Prefers Three Dimensions}

\section{Table of Runs}
\label{sec:runs}

\section{Resolution and Reynolds number study}
\label{sec:resolution}

\section{Asymptotic Calculation}
\label{sec:asymp}
Define the domain as $ - d/2 < x < d/2$. Make it symmetric so to make the shear simpler. The shear has no net. Otherwise it means we are not exactly in the rotating frame we thought we were. 
\Beq
\int_{-d/2}^{d/2} S x \text{d} x  \ = \ 0
\Eeq
This is very important later. 

The full linear equations:
\Beq
\pd{t} u + S x \pd{y} u - f v + \pd{x} p - B \pd{z} b_{x} &=& 0\\
\pd{t} v + S x \pd{y} v + (S+f) u + \pd{y} p - B \pd{z} b_{y} &=& 0\\
\pd{t} w + S x \pd{y} w + \pd{z} p - B \pd{z} b_{z} &=& 0 \\
\pd{x} u + \pd{y} v + \pd{z} w  &=& 0\\
\pd{t} b_{x} + S x \pd{y} b_{x} - B \pd{z} u &=& 0\\
\pd{t} b_{y} + S x \pd{y} b_{y} - B \pd{z} v - S b_{x}  &=& 0\\
\pd{t} b_{z} + S x \pd{y} b_{z} - B \pd{z} w &=& 0
\Eeq

Assume all variable take the form
\Beq
g = \hat{G}(x) e^{ i ( \omega t + k z + \ell y ) } 
\Eeq

Define the ``parameter''
\Beq
\omega_{S} \ \equiv \ \omega + S x \ell 
\Eeq

We can use equations (2)--(7) to find all amplitudes in terms of $\hat{U}(x)$ and $\hat{U}'(x)$. 
\Beq
\hat{V} &=& \frac{i \left(k^2 \hat{U}(x) \left(B^2 k^2 S-(f+S) \omega
   _S^2\right)+\ell  \omega _S \hat{U}'(x) \left(B^2 k^2-\omega
   _S^2\right)\right)}{\left(k^2+\ell ^2\right) \omega _S
   \left(B^2 k^2-\omega _S^2\right)}\\
\hat{W} &=& -\frac{i k \left(\ell  \hat{U}(x) \left(B^2 k^2 S-(f+S) \omega
   _S^2\right)+\omega _S U'(x) \left(\omega _S^2-B^2
   k^2\right)\right)}{\left(k^2+\ell ^2\right) \omega _S
   \left(B^2 k^2-\omega _S^2\right)}\\
   \hat{P} &=& -\frac{i \left(\ell  \hat{U}(x) \left(B^2 k^2 S-(f+S) \omega
   _S^2\right)+\omega _S U'(x) \left(\omega _S^2-B^2
   k^2\right)\right)}{\left(k^2+\ell ^2\right) \omega _S^2}\\
   \hat{B}_{x} &=& \frac{B k \hat{U}(x)}{\omega _S}\\
   \hat{B}_{y} &=& -\frac{i B k \left(\hat{U}(x) \left(B^2 k^2 S \ell ^2+\omega _S^2
   \left(f k^2-S \ell ^2\right)\right)+\ell  \omega _S
   U'(x) \left(\omega _S^2-B^2
   k^2\right)\right)}{\left(k^2+\ell ^2\right) \omega _S^2
   \left(B^2 k^2-\omega _S^2\right)}\\
   \hat{B}_{z} &=& -\frac{i B k^2 \left(\ell  \hat{U}(x) \left(B^2 k^2 S-(f+S)
   \omega _S^2\right)+\omega _S U'(x) \left(\omega _S^2-B^2
   k^2\right)\right)}{\left(k^2+\ell ^2\right) \omega _S^2
   \left(B^2 k^2-\omega _S^2\right)}
\Eeq

We can then substitute everything into equation (1) to get a second-order equation for $\hat{U}$. 

When we do this, we get something of the form 
\Beq
\alpha(x) \hat{U}''(x) + \beta(x) \hat{U}'(x) + \gamma(x) \hat(U)(x) \ = \ 0
\Eeq
where
\Beq
\alpha(x) &=& \frac{i \left(B^2 k^2-\omega _s^2\right)}{\left(k^2+\ell
   ^2\right) \omega _s} \\ 
   \beta(x) &=& -\frac{2 i B^2 k^2 S \ell }{\left(k^2+\ell ^2\right) \omega
   _s^2} \\
   \gamma(x) &=& \frac{i \left(\omega _s^4 \left(\frac{f^2
   k^2}{\left(k^2+\ell ^2\right) \left(B^2 k^2-\omega
   _s^2\right)}+1\right)-k^2 \omega _s^2 \left(B^2+\frac{f
   S}{k^2+\ell ^2}\right)+\frac{2 B^2 k^2 S^2 \ell
   ^2}{k^2+\ell ^2}\right)}{\omega _s^3}
\Eeq

We can eliminate the first-order term via
\Beq
\hat{U}(x) \ = \ \chi(x) \psi(x) 
\Eeq
if 
\Beq
\chi'(x) \  = \ -\frac{\beta (x) \chi (x)}{2 \alpha (x)}
\Eeq
\Beq
\psi''(x) + \Phi(x) \psi(x) \ = \ 0
\Eeq
\Beq
\Phi(x)  \ = \ -\frac{-2 \beta (x) \alpha '(x)+2 \alpha (x) \left(\beta'(x)-2 \gamma (x)\right)+\beta (x)^2}{4 \alpha (x)^2}
\Eeq
Putting everything together 
\Beq
\Phi(x) \ = \ \tfrac{k^2 \omega _S^2 \left(2 B^2 \left(k^2+\ell
   ^2\right)+f (f+S)\right)-B^2 k^2 \left(B^2 k^2
   \left(k^2+\ell ^2\right)+S \left(f k^2-S \ell
   ^2\right)\right)-\left(k^2+\ell ^2\right) \omega
   _S^4}{\left(\omega _S^2-B^2 k^2\right){}^2}
\Eeq

Now we want to rescale everything in terms of a bookkeeping parameter, $\eps$.
\Beq
S  \ \to \  - \frac{\pi ^2 B^2}{d^2 f} ( 1 + \eps^{2} R) 
\Eeq 
\Beq
\omega \ \to \ \eps^{2} \omega , \quad k \ \to \ \eps k, \quad \ell \ \to \ \eps^{2} \ell 
\Eeq
Also 
\Beq
q  \ \equiv  \ \frac{\pi ^2 B^2}{d^2 f^{2}}
\Eeq

Expanding to $\mathcal{O}(\eps^{2})$, 
\Beq
\Phi(x) \ = \ \frac{\pi^{2}}{d^{2}} + \frac{\epsilon ^2 \left(-B^4 k^4+B^2 f^2 q \left(k^2 R+q
   \ell ^2\right)+f^2 (q+1) (\omega -f q x \ell
   )^2\right)}{B^4 k^2}
\Eeq

Now do the perturbations 

\Beq
\psi_{0}''(x) + \frac{\pi^{2}}{d^{2}}  \psi_{0}(x) \ = \ 0 
\Eeq
\Beq
\psi_{2}''(x) + \frac{\pi^{2}}{d^{2}}  \psi_{2}(x) + \Phi_{2}(x)\psi_{0}(x)\ = \ 0 
\Eeq

Remember we are on $-d/2 < x < d/2$ Therefore 
\Beq
\psi_{0}(x) \ = \ \cos(\pi x / d)
\Eeq
This satisfies the equation and boundary condition. The solvability condition for $\psi_{2}(x)$ is 

\Beq
\int_{-d/2}^{d/2} \Phi_{2}(x)\psi_{0}(x)^{2} \text{d} x \ = \ 0.
\Eeq


This works out to 
\Beq
\left(\pi ^2 B^2+d^2 f^2 \right) \omega^{2} - B^4 k^2 \left(d^2 k^2-\pi ^2 R\right)   \ + \ \frac{\pi ^2 B^4 \ell ^2}{12}  \left(\left(6+\pi ^2\right) q+\pi ^2-6\right) \ = \ 0
\Eeq



\Beq
\omega^{2} \ = \ \frac{q\,B^{2}}{(q+1)} \left[ k_z^2 \left(\frac{d^2}{\pi^{2}} k_z^2-R\right) \ - \ \frac{\left(6+\pi ^2\right) q+\pi
   ^2-6}{12 } \, k_{y}^{2}  \right] \ + \ ...
\Eeq


\end{document}
