\documentclass[aps,prl,preprint
,superscriptaddress]{revtex4-1}
\newcommand{\SSC}{S/S_{c}}
\newcommand{\mnras}{Monthly Notices of the Royal Astronomical Society}
\newcommand{\apjs}{The Astrophysical Journal Supplement Series}
\usepackage{graphicx}
\usepackage{amssymb}
\usepackage{amsmath}

% You should use BibTeX and apsrev.bst for references
% Choosing a journal automatically selects the correct APS
% BibTeX style file (bst file), so only uncomment the line
% below if necessary.
%\bibliographystyle{apsrev4-1}

\newcommand\Beq{\begin{eqnarray}} 
\newcommand\Eeq{\end{eqnarray}}

\newcommand{\Eq}[1]{Eq.~(\ref{#1})}
\newcommand{\Eqs}[2]{Eqs.~(\ref{#1})--(\ref{#2})}
\newcommand{\eq}[1]{eq.~(\ref{#1})}
\newcommand{\eqs}[2]{eqs.~(\ref{#1})--(\ref{#2})}

\newcommand{\Prob}[2]{\mathbb{P}[\, #1 \, | \, #2 \, ] }
\newcommand{\Pro}[1]{\mathbb{P}[\, #1 \, ] }

\newcommand{\inv}[1]{\frac{1}{#1}}
\newcommand{\pd}[1]{\partial_{#1}}

\newcommand{\Ra}{R}
\newcommand{\Ve}{\mbox{\textit{Ve}}}
\newcommand{\eps}{\varepsilon}
\newcommand{\del}{\delta}
\newcommand{\e}[1]{\eps^{#1}}
\newcommand{\f}{2\Omega}

\newcommand{\n}{\hat{n}}


\newcommand{\dat}{\cdot}
\newcommand{\dd}[1]{\,\mathrm{d}{#1}}
\newcommand{\cc}{\mathrm{c.c.}}
\renewcommand{\vec}[1]{\boldsymbol{#1}}
\newcommand{\grad}{\nabla}
\renewcommand{\div}{\nabla \dat}
\newcommand{\Tr}{\text{Tr}}


\newcommand{\pren}[1]{ \left(#1\right) }
\newcommand{\brac}[1]{\left[\,#1\,\right]}

\newcommand{\ave}[1]{\big<#1\big>}

\renewcommand{\l}{\ell}

\newcommand{\txt}[1]{\quad \text{#1} \quad }

\begin{document}

% Use the \preprint command to place your local institutional report
% number in the upper righthand corner of the title page in preprint mode.
% Multiple \preprint commands are allowed.
% Use the 'preprintnumbers' class option to override journal defaults
% to display numbers if necessary
%\preprint{}

%Title of paper
\title{Supplemental Information for: The Magnetorotational Instability Prefers Three Dimensions}

\section{Numerical Methods}
\label{sec:methods}

For both the ideal and non-ideal MHD equations, we solve the eigenvalue problem in the $x$ direction using the EigenValueProblem mode in \emph{Dedalus} for a grid of $N_y \times N_z$ modes in the $y$ and $z$ directions.
For most of our runs, we use a targeted, sparse eigenvalue solver solving for the 15 modes closest to a guess for the maximum growth rate.
We use the ideal 2D growth rate as input for the smallest $k_y > 0$ mode at each $k_z$ and then use the output from each previous $k_y$ as an input guess for the next mode.
Our solver is parallelized over the $k_y, k_z$ modes.
We have confirmed that dense solvers retrieve identical results.
For the spectrum in figure 1 of the main text, we used a dense eigenvalue solver.

In order to ensure our results are converged, we have repeated runs at $n_x=256$ Chebyshev modes as well as doubling the number of modes in $y$ and $z$.

\begin{tabular}{cccccccccccccc}
\textbf{Run} & \textbf{$\SSC$} & \textbf{$\nu$} & \textbf{$\eta$} & \textbf{$q$} & \textbf{$N_x$} & \textbf{$N_{k_y}$} & \textbf{$N_{k_z}$} & \textbf{Sparse/Dense}& \\
1  &   1.02 & $10^{-5}$ & $10^{-5}$ & 0.75 & 128 & 200 & 200 & sparse & resolution study\\
2  &   1.02 & $10^{-5}$ & $10^{-5}$ & 0.75 & 256 & 200 & 200 & sparse & \\
3  &   1.02 & $10^{-5}$ & $10^{-5}$ & 0.75 & 128 & 200 & 200 & dense  & \\
4  &   1.02 & $10^{-5}$ & $10^{-5}$ & 0.75 & 128 & 512 & 512 & sparse & \\
5  &   1.02 & $10^{-5}$ & $10^{-5}$ & 0.75 & 128 & 100 & 100 & sparse & \\
6  &   0.2  & $10^{-5}$ & $10^{-5}$ & 0.75 & 128 & 200 & 200 & sparse & $\SSC$ variation\\
7  &   0.3  & $10^{-5}$ & $10^{-5}$ & 0.75 & 128 & 200 & 200 & sparse & \\
8  &   0.4  & $10^{-5}$ & $10^{-5}$ & 0.75 & 128 & 200 & 200 & sparse & \\
9  &   0.5  & $10^{-5}$ & $10^{-5}$ & 0.75 & 128 & 200 & 200 & sparse & \\
10 &   0.64 & $10^{-5}$ & $10^{-5}$ & 0.75 & 128 & 200 & 200 & sparse & \\
11 & 1.020  & $10^{-5}$ & $10^{-5}$ & 0.75 & 128 & 200 & 200 & sparse & \\
12 & 1.44   & $10^{-5}$ & $10^{-5}$ & 0.75 & 128 & 200 & 200 & sparse & \\
13 & 1.75   & $10^{-5}$ & $10^{-5}$ & 0.75 & 128 & 200 & 200 & sparse & \\
14 & 1.891  & $10^{-5}$ & $10^{-5}$ & 0.75 & 128 & 200 & 200 & sparse & \\
15 & 2.     & $10^{-5}$ & $10^{-5}$ & 0.75 & 128 & 200 & 200 & sparse & \\
16 & 2.01   & $10^{-5}$ & $10^{-5}$ & 0.75 & 128 & 200 & 200 & sparse & \\
17 & 2.015  & $10^{-5}$ & $10^{-5}$ & 0.75 & 128 & 200 & 200 & sparse & \\
18 & 2.031  & $10^{-5}$ & $10^{-5}$ & 0.75 & 128 & 200 & 200 & sparse & \\
19 & 2.05   & $10^{-5}$ & $10^{-5}$ & 0.75 & 128 & 200 & 200 & sparse & \\
20 & 2.1    & $10^{-5}$ & $10^{-5}$ & 0.75 & 128 & 200 & 200 & sparse & \\
21 & 2.25   & $10^{-5}$ & $10^{-5}$ & 0.75 & 128 & 200 & 200 & sparse & \\
22 & 2.5    & $10^{-5}$ & $10^{-5}$ & 0.75 & 128 & 200 & 200 & sparse & \\
23 & 4      & $10^{-5}$ & $10^{-5}$ & 0.75 & 128 & 200 & 200 & sparse & \\
24 & 1.02   & $10^{-6}$ & $10^{-6}$ & 0.75 & 128 & 200 & 200 & sparse & Reynolds number study\\
25 & 1.02   & $10^{-4}$ & $10^{-4}$ & 0.75 & 128 & 200 & 200 & sparse & \\
26 & 1.02   & $10^{-3}$ & $10^{-3}$ & 0.75 & 128 & 200 & 200 & sparse & \\
27 & 1.02   & $10^{-2}$ & $10^{-2}$ & 0.75 & 128 & 200 & 200 & sparse & \\
28 & 1.02   & $10^{-5}$ & $10^{-5}$ & 0.1  & 128 & 200 & 200 & sparse & Low Rossby\\ 
29 & 1.02   & $10^{-6}$ & $10^{-2}$ & 0.1  & 128 & 200 & 200 & sparse & Liquid metal case\\ 
\end{tabular}
%\label{simulations}



\section{Resolution and Reynolds number study}
\label{sec:resolution}

\section{Asymptotic Calculation}
\label{sec:asymp}
Define the domain as $ - d/2 < x < d/2$. Being in the rotating frame means the shear has no net
\Beq
\int_{-d/2}^{d/2} V_{y}(x) \text{d} x  \ = \ 0, \quad \implies  \quad V_{y}(x) \ = \ S\, x
\Eeq
The full linear ideal equations are:
\Beq\label{u-eq}
\pd{t} u + S x \pd{y} u - f v + \pd{x} p - B_{0} \pd{z} b_{x} &=& 0\\
\pd{t} v + S x \pd{y} v + (f+S) u + \pd{y} p - B_{0} \pd{z} b_{y} &=& 0\label{v-eq} \\
\pd{t} w + S x \pd{y} w + \pd{z} p - B_{0} \pd{z} b_{z} &=& 0 \\
\pd{x} u + \pd{y} v + \pd{z} w  &=& 0\\
\pd{t} b_{x} + S x \pd{y} b_{x} - B_{0} \pd{z} u &=& 0\\
\pd{t} b_{y} + S x \pd{y} b_{y} - B_{0} \pd{z} v - S b_{x}  &=& 0\\
\pd{t} b_{z} + S x \pd{y} b_{z} - B_{0} \pd{z} w &=& 0 \label{bz-eq}
\Eeq
All variables take the form
\Beq
g = \hat{G}(x) e^{ i ( \omega t + k z + \ell y ) } 
\Eeq
We solve the ideal eigenvalue problem in \textit{Dedalus} in the above formulation. To make analytical progress, define the frequency ``parameters'' 
\Beq
\omega_{S}(x) \ \equiv \ \omega + S x \ell, \quad \omega_{A}  \ \equiv \ B_{0} k.
\Eeq
We can use equations (\ref{v-eq}--\ref{bz-eq}) to find all amplitudes in terms of $\hat{U}(x)$ and $\hat{U}'(x)$. 
\Beq
\hat{V}(x) &=& \frac{i \left(\ell 
   \hat{U}'(x) + k^{2} \frac{ \left((f+S) \omega
   _S^2-S \omega _A^2\right)}{\omega _S
   \left(\omega _S^2-\omega _A^2\right)} \hat{U}(x)\right)}{k^2+\ell ^2} \\
\hat{W}(x) &=& \frac{i k \left(\hat{U}'(x)- \frac{ \left((f+S) \omega
   _S^2-S \omega _A^2\right)}{\omega _S
   \left(\omega _S^2-\omega _A^2\right)}  \ell 
   \hat{U}(x)\right)}{k^2+\ell ^2} \\
   \hat{P}(x) &=& \frac{i \left(\omega _A^2-\omega _S^2\right)
   \left(\hat{U}'(x) -  \ell \frac{ \left((f+S) \omega
   _S^2-S \omega _A^2\right)}{\omega _S
   \left(\omega _S^2-\omega _A^2\right)}
   \hat{U}(x)\right)}{\left(k^2+\ell ^2\right)
   \omega _S} \\
   \hat{B}_{x}(x) &=& \frac{\omega _A \hat{U}(x)}{\omega _S} \\
   \hat{B}_{y}(x) &=&\frac{\omega _A \hat{V}(x)}{\omega _S} -\frac{i S \omega _A \hat{U}(x)}{\omega _S^2} \\
   \hat{B}_{z}(x) &=& \frac{\omega _A \hat{W}(x)}{\omega _S}
   \Eeq
Substitute everything into equation (\ref{u-eq}) to get a second-order equation for $\hat{U}(x)$ of the general form 
\Beq
\alpha(x) \hat{U}''(x) + \beta(x) \hat{U}'(x) + \gamma(x) \hat{U}(x) \ = \ 0
\Eeq
where
\Beq
\alpha(x) &=& \omega _S^2-\omega _A^2 \\ 
   \beta(x) &=& \frac{2 S \ell  \omega _A^2}{\omega _S} \\
   \gamma(x) &=& \omega _A^2 \left(\frac{f^2 k^2}{\omega
   _S^2-\omega _A^2}-\frac{2 S^2 \ell
   ^2}{\omega _S^2}+k^2+\ell ^2\right)+f k^2
   (f+S)-\left(k^2+\ell ^2\right) \omega _S^2
\Eeq


We can eliminate the first-order term via
\Beq
\hat{U}(x) \ = \ \chi(x) \psi(x) \quad \text{where} \quad \chi'(x) \  = \ -\frac{\beta (x) \chi (x)}{2 \alpha (x)}
\Eeq
\Beq
-\psi''(x) + \Phi(x) \psi(x) \ = \ 0 \label{Schr-eq}
\Eeq
\Beq
\Phi(x)  \ = \ \frac{-2 \beta (x) \alpha '(x)+2 \alpha (x)
   \left(\beta '(x)-2 \gamma (x)\right)+\beta
   (x)^2}{4 \alpha (x)^2}
\Eeq
Putting everything together 
\Beq
\Phi(x) \ = \ \frac{S  \left(f k^2-S \ell
   ^2\right) \omega _A^2 -f k^2 (f+S) \omega
   _S^2}{\left(\omega _A^2-\omega
   _S^2\right){}^2} + k^{2} + \ell^{2}
\Eeq
If $\omega_{S}=-i \gamma+Sx\ell$, with $Re(\gamma)\neq0$ then the denominator of $\Phi(x)$ cannot vanish. 
We want to solve the Schr\"{o}dinger-type equation~(\ref{Schr-eq}) with the boundary conditions 
\Beq
\psi(x=\pm d) = 0.
\Eeq
We solve perturbatively near the critical values for the 2D  instability. We rescale each parameters in terms of a bookkeeping parameter, $\eps$.
\Beq
S  \ \to \  - \frac{\pi ^2 B_{0}^2}{d^2 f} ( 1 + \eps^{2} R), \quad \omega \ \to \ \eps^{2} \omega , \quad k \ \to \ \eps k, \quad \ell \ \to \ \eps^{2} \ell
\Eeq 
Also 
\Beq
q  \ \equiv  \ \frac{\pi ^2 B_{0}^2}{d^2 f^{2}}
\Eeq
Expanding to $\mathcal{O}(\eps^{2})$, 
\Beq
\Phi(x) \ = \ \frac{\pi^{2}}{d^{2}} + \eps^{2} \frac{B_{0}^2 f^2 q \left(k^2 R+q
   \ell ^2\right) - B_{0}^4 k^4 +f^2 (q+1) (\omega -f q x \ell
   )^2  }{B_{0}^4 k^2}
\Eeq
For leading order 
\Beq
\psi_{0}''(x) + \frac{\pi^{2}}{d^{2}}  \psi_{0}(x) \ = \ 0  \quad \implies \quad \psi_{0}(x) \ = \ \cos(\pi x / d).
\Eeq
The next order is
\Beq
\psi_{2}''(x) + \frac{\pi^{2}}{d^{2}}  \psi_{2}(x) + \Phi_{2}(x)\psi_{0}(x)\ = \ 0.
\Eeq
The solvability condition for $\psi_{2}(x)$ is 
\Beq
\int_{-d/2}^{d/2} \Phi_{2}(x)\psi_{0}(x)^{2} \text{d} x \ = \ 0.
\Eeq
This implies 
\Beq
-\omega^{2} \ = \ \frac{q\,B_{0}^{2}}{(q+1)} \left[ k_z^2 \left(R-\frac{d^2}{\pi^{2}} k_z^2\right) \ + \ \frac{\left(6+\pi ^2\right) q+\pi
   ^2-6}{12 } \, k_{y}^{2}  \right] \ + \ ...
\Eeq


\end{document}
