\documentclass[letterpaper, 12pt]{article}

% Imports
\usepackage{amsmath, amssymb}
\usepackage{mathtools}
\usepackage{graphicx}

% Layout
\usepackage[margin=1in]{geometry}
\usepackage{setspace}
\setlength{\parindent}{0em}
\setlength{\parskip}{1em}

% Body font
\usepackage{soul, color}
\usepackage[colorlinks, allcolors=blue]{hyperref}
\usepackage{newtxtext}

% Reviews
\usepackage{framed}
\newenvironment{review}{\vspace{1em}\begin{leftbar}}{\end{leftbar}}

\begin{document}

Dear Professor Tobias,

We thank the referees for constructive reports. We have made all of the changes they suggested. In addition, we added section 3c detailing an asymptotic confirmation of the 3D to 2D transition reported numerically in section 3a to at least three significant figures. If this section is too much of an addition at this stage, we can relegate it to an appendix or remove it.

Please see our responses in line below.

\begin{review}
Comments to the Author(s)
This is an interesting manuscript that reveals a new and important result regarding the magnetorotational instability (MRI) in a radially confined geometry such as a Taylor-Couette experiment between differentially rotating cylinders. The authors show that 3D (non-axisymmetric) modes can be preferred to 2D (axisymmetric) ones, especially near the onset of the 2D instability. This might be important for the outcome of MRI experiments such as the one at Princeton, and it also has ramifications for dynamo action. The authors obtain good agreement, where expected between their numerical calculations of unstable modes and their asymptotic analysis.

The manuscript is mostly well written and the figures are well presented (although some are too small in the current format - in particular, Fig. 3 is hard to read).

I recommend publication in Proceedings A provided the following matters can be addressed.

1. It would be helpful to explain that "three-dimensional" means "non-axisymmetric" and "two-dimensional" means "axisymmetric" (but with three-component velocity and magnetic fields). (Of course, "axisymmetric" is to be interpreted as "y-independent" in this local model.)
\end{review}
We have fixed this.

\begin{review}  
2. Section 1: Perhaps clarify that the "finite-channel cutoff" is due to the finite width, not height.
\end{review}
Fixed.

\begin{review}
3. Section 1: What does "weak-shear" mean here: weak compared to what?  
\end{review}

Fixed

\begin{review}
4. Section 2: The notation could be defined more clearly by saying what v, b and j represent. We certainly need to know that f = 2 Omega.
\end{review}

We are unclear what the referee refers to here; these are already in the text.

\begin{review}
5. Section 2: The channel is periodic in z, but it seems that the periodicity length is unspecified, so that any value of the wavenumber $k_z$ is permissible. Please clarify this.
\end{review}
  
Fixed.

\begin{review}
6. Section 2: Personally, I find it confusing to have the symbol q denoting the Rossby number when (as said in the footnote) in most of the MRI literature q denotes twice the Rossby number. Would it not be better to use a different symbol for the Rossby number?
\end{review}

Fixed

\begin{review}
7. Section 2: I believe that q > 1 (rather than q = 1, as stated) corresponds to Rayleigh-unstable shear. (Are we to assume throughout that q > 0, i.e. S < 0?)
\end{review}
Fixed

\begin{review}
8. Around eq. 2.6: Please clarify the meaning of "in two dimensions" (i.e. axisymmetric, or y-independent, perturbations). Is there a convenient reference for eq. 2.6, as the result is not obvious? I believe it comes from taking the limit $k_z d \ll 1$, so the periodicity length in z has to be very long.
\end{review}

Fixed.


\begin{review}
9. Section 2: An important point is that the system of units has not been clarified and the relevant dimensionless parameters have not all been identified. Everything is expressed in dimensional terms until we are told that nu = eta = 1e-5. In what units are these values expressed?
  \end{review}

We have clarified this point while simultaneously addressing point 16.

\begin{review}
10. Section 2: I thought initially that there was a missing, and unspecified, dimensionless parameter such as $B_0 / f d$, which is a measure of the field strength. I see now that this can be determined from the given values of $q$ and $S/S_c$, but it might be better to explain this in the text.
\end{review}

We have done so.

\begin{review}
11. Figure 1 caption: Presumably "cray" should be "grey".  
\end{review}


Fixed

\begin{review}
12. Before eq. 3.1: Presumably "in when" should be "when".  
\end{review}

Fixed


\begin{review}
13. Before eq. 3.9: "2nd order" should probably be "2nd order in x".  
\end{review}

Fixed


\begin{review}
14. Eq. 3.9: Why are $k_z$ and $k_x$ renamed $k$ and $l$ here?  
\end{review}

We have corrected this.


\begin{review}
15. Section 3(c): It seems to me that the prevalence of axisymmetric modes in previous numerical simulations may have more to do with the radial boundary conditions than with the degree of supercriticality.
\end{review}

A note was added to this effect.


\begin{review}
16. Section 3(e): It would have been better to define Re and Rm back in Section 2. Again, the units in which eta and nu are given are not specified. There are a couple of typos in the first sentence: "as series" should be "a series", and "with the" should be "with".
\end{review}

Typos fixed and Reynolds numbers moved upwards.


\begin{review}
17. I disagree that there is only one previous study of non-turbulent MRI dynamos; indeed, reference [15] and related papers provide examples of this.  
\end{review}

This is completely correct; our original paper was in error. We meant to express that there are no other \emph{linear} MRI dynamos, not \emph{laminar} ones!



\begin{review}
Referee: 2

Comments to the Author(s)
Thank you very much for sending your article! I find the subject, three-dimensional (3d) modes of the magneto-rotational instability (MRI), a very interesting aspect that so far has received much less attention than the corresponding two-dimensional (2d) modes. Your analysis is very well suited to the task, the results are interesting and presented clearly and in detail and set into the context of conditions in stars and lab experiments . Hence I recommend the article for publication. I have, however, a few minor points to
raise. The primary one is a question regarding the range of applicability of the 3d results, in particular in stars. You vary the ratio between the shear, S, and the critical shear value, $S_c$, given in equation (2.6), from values below unity to a maximum of 4. You note that for $S/S_c >~ 2.05$, the linear dynamics should be dominated by 2d
modes. $S_c$ involves, among other quantities, the square of the magnetic field strength, which can vary by (or is uncertain to) quite large factors. Thus, even moderate variations of the field may change $S_c$ such that $S/S_c$ no longer favours 3d modes. My question is, thus, if this apparent need for what might be considered fine-tuning limits the applicability of the results to, e.g., stars.
\end{review}
Thank you for pointing this out. We have added a cautionary note regarding this point.


\begin{review}
Apart from that, I have only small points:
- Section 3(a): the text mentions $S/S_c = 1.02$, but Fig. 1 has 1.002 and 1.2. Also, at the end of the paragraph: "does not intercept the x-axis"; one might put "$k_z$-axis" to avoid confusion with the axes of the simulation box.
- Section 3(f), after the equation: "rms is denotes...": remove one of the two verbs.
- Also 3(f): does the integral of the EMF over x vanish for reasons of symmetry?
\end{review}

We have fixed all of these.

\vspace{1em}

Sincerely, \\
Jeffrey S. Oishi, Geoffrey M. Vasil, Morgan Baxter, Andrew Swan, Keaton J. Burns, Daniel Lecoanet, \& Benjamin P. Brown
\end{document}

%%% Local Variables:
%%% ispell-local-dictionary: "british"
%%% End:

%  LocalWords:  Tobias magnetorotational Couette axisymmetric Rossby
%  LocalWords:  wavenumber eq nd supercriticality rms Oishi Vasil
%  LocalWords:  Lecoanet
